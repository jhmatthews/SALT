{\it In the last few weeks, our team has used a Monte Carlo radiative transfer
code to show that accretion disk winds can have a profound impact on the optical spectra of Cataclysmic Variables (CVs). Here we propose a set of spectroscopic observations of three CVs, 
with our motivation for submitting past the deadline being due to the following key points:

\begin{itemize}
\item The results which have led to this proposal have been {\bf very} recently obtained.
\item Due to the brightness of the objects, the observations are extremely cheap; we estimate a combined total of 30 minutes for the three objects in question to obtain signal-to-noise ratios of 30+. 
\item The data will be made publicly available, and serve as something of an online `atlas' for the CV community.
\item The spectra are required because broadband, single epoch spectroscopy of the systems in question is not currently available to the required spectral resolution.
\end{itemize}
}
\bigskip



Cataclysmic variables are systems in which a white dwarf accretes matter from a donor
star via Roche-lobe overflow. In non-magnetic nova-like systems (NMNLs) this accretion
is mediated by an accretion disk which forms around the white dwarf, and emits in the optical
and ultraviolet. NMNLs act as the perfect laboratory for accretion physics 
and testing of the `simple' disk model proposed by Shakura \& Sunyaev (1972), with one
specific example being the testing of the predicted $T\propto R^{-3/4}$ temperature 
profile with eclipse mapping (Rutten, van
Paradijs \& Tinbergen 1992).
\bigskip
%% be more specific ''in many ways'' -> goes
%% mention T~R3/4 ratios, eclipse mapping? 'simple' disk


For over three decades, it has been known that winds emanating from the accretion disk
are important in shaping the ultraviolet spectra of CVs (Heap 1978), 
the most spectacular evidence being the P-Cygni like profiles of resonance lines such as 
C${\textsc {iv}}$ (see e.g. Cordova \& Mason 1982). 
However, the extent to which winds influence optical spectra is not known, and even their origin 
and driving mechanism remains unclear (Drew \& Proga 2000). 
Answering these questions has far reaching implications, as
disk winds are of astrophysical importance across many orders of magnitudes in mass.
They are proposed as an important mechanism for AGN feedback (Silk \& Rees 1998) and shaping 
the spectra of Quasars (Weymann et al. 1991), and understanding them is vital to test
unification of accreting objects.
%% more here. Why are winds important? far reaching implications, big picture

\bigskip
Our recent Monte Carlo radiative transfer simulations (Matthews et al., in prep) expand on the work of Long \& Knigge (2002) by incorporating line transfer techniques suggested by Lucy (2002, 2003). Excitingly, these improvements have enabled us to show that the same outflow models used to explain the ultraviolet features seen in CVs also have a significant impact on optical features. In particular, we find that recombination lines in Hydrogen and Helium can be produced by a disk wind, and the same wind geometry can `fill in' the Balmer absorption edge that has thus far been present in CV models (but not observations; see e.g. Knigge \& Drew 1997). An example synthesized spectrum can be seen in Figure 1.
%% show spectra

\bigskip
We propose a spectroscopic study of three classic high-state systems at different inclinations. {\bf RW Sextantis, IX Velorium} and {\bf UU Aquarii} are all simple disk CVs with high accretion rates, and hence also have potential for mass loss. At inclinations of $\sim30^\circ$, $\sim60^\circ$ and $\sim80^\circ$ respectively these three objects provide opportunities to probe spectra across the full range of viewing angles. 
%% why are observations needed? why do we need better spectra? Specifics.
%% why use SALT?

\bigskip
These observations are essential for validating our results and will provide a useful resource to the community. They are required because sufficient quality spectra of NMNLs at varying inclinations are not available, and SALT's spectral capabilities provide the perfect opportunity to observe these objects.
In symbiosis with our modeling, the analysis of these spectra will help us to draw conclusions about the nature of the Balmer jump, the recombination lines of H and He and the continuity between disk atmosphere and disk wind. This will provide answers of astrophysical importance from a relatively modest time investment, and will also result in prompt publication for maximum benefit to the community.